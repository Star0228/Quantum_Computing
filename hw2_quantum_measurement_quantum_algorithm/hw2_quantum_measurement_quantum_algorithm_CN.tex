\documentclass[11pt]{article}
\usepackage{amsmath,amssymb,enumitem,algorithm,algpseudocode,graphicx}
\usepackage[UTF8]{ctex}
\usepackage{enumerate}
\usepackage[braket]{qcircuit}
\parindent=22pt
\parskip=3pt
\oddsidemargin 18pt \evensidemargin 0pt
\leftmargin 1.5in
\marginparwidth 1in \marginparsep 0pt \headsep 0pt \topskip 20pt
\textheight 225mm \textwidth 148mm
\renewcommand{\baselinestretch}{1.15}
\begin{document}
\title{{\bf 理论作业二 \quad 量子测量与量子算法}}
\author{王晓宇 \quad 3220104364}
\date{\today}
\maketitle

\begin{tabular*}{13cm}{r}
\hline
\end{tabular*}

\vskip 0.3 in

{\bf 1.} 假设有初始化为 $|1\rangle$ 态的量子寄存器若干,给出分别使用酉算子 $H$、$X$、$T$、$S$ 进行测量的结果。

\begin{enumerate}[i.]
    \item 使用 $H$ 算子进行测量。
        \[|1\rangle \stackrel{H}\longrightarrow  \frac{1}{\sqrt{2}}\left( |0\rangle - |1\rangle\right)\]
    \item 使用 $X$ 算子进行测量。
        \[|1\rangle \stackrel{X}\longrightarrow  |0\rangle\]
    \item 使用 $T$ 算子进行测量。
        \[T = 
        \begin{bmatrix}
            1 & 0 \\
            0 & e^{i\frac{\pi}{4}}    
        \end{bmatrix}\]
        \[|1\rangle \stackrel{T}\longrightarrow  e^{i\frac{\pi}{4}}|1\rangle \]
    \item 使用 $S$ 算子进行测量。
        \[S = 
        \begin{bmatrix}
            1 & 0 \\
            0 & i    
        \end{bmatrix}\]
        \[|1\rangle \stackrel{S}\longrightarrow  i |1\rangle \]
\end{enumerate}


\vskip 0.3 in

{\bf 2.} 证明 Grover 算法中的算子 $G$ 每次作用时使量子态向 $|\beta\rangle$ 方向旋转角度 $\theta$。







初始状态为$|\psi\rangle = \sqrt{\frac{N-M}{N}}|\alpha\rangle + \sqrt{\frac{M}{N}} |\beta\rangle= \cos(\frac{\theta}{2}) |\alpha\rangle + \sin(\frac{\theta}{2}) |\beta\rangle$

其中N为待检验的解个数,M为可行解的数量。

这里为了简便表示$|\psi\rangle$,我们将系数表示为$\cos(\frac{\theta}{2})$和$\sin(\frac{\theta}{2})$,其中$\frac{\theta}{2}$为初始态与正解轴$\alpha$之间的夹角。


我们将Grover算法中的算子$G$作用按照课上所讲分为两个部分,即Oracle和Combined(2、3、4步骤结合的酉矩阵)。

\begin{enumerate}[i.]
    \item Orcale
    作用:改变正解相位
       \[|x\rangle \stackrel{Orcale}\longrightarrow  (-1)^{f(x)}|x\rangle\]
    平面作用:以$|\alpha\rangle$为轴做对称操作,在$|\alpha\rangle$上的投影不变,即$|\alpha\rangle$不变,$|\beta\rangle$上的投影翻转。
    即:
    \[|x\rangle = p |\alpha\rangle + q |\beta\rangle \stackrel{Orcale}\longrightarrow  p |\alpha\rangle - q|\beta\rangle\]
    \item Combined
    我们这里表示$|x\rangle$时更换基:$|x\rangle = p |\psi\rangle + q |\psi\rangle_\perp$,其中$|\psi\rangle_{\perp}$为与$|\psi\rangle$正交的向量。
    作用:将$|\psi\rangle$向$|\beta\rangle$方向旋转
    \[|x\rangle \stackrel{Combined}\longrightarrow \left(2|\psi\rangle\langle\psi| - I\right)|x\rangle\]   
    平面作用:以$|\psi\rangle$为轴做对称操作
    \[|x\rangle =  p |\psi\rangle + q |\psi\rangle_\perp \stackrel{Combined}\longrightarrow   p |\psi\rangle - q |\psi\rangle_\perp\]
\end{enumerate}


\textbf{我们证明的目标是证明Grover算法中的算子$G$每次作用时使量子态向$|\beta\rangle$方向旋转角度$\theta$。
这里可以用数学归纳法证明:}
\begin{enumerate}[(1)]
    \item 归纳奠基
    
    证明$G$算子作用到初态上时成立:
    \[|x\rangle = |\psi\rangle = \cos(\frac{\theta}{2}) |\alpha\rangle + \sin(\frac{\theta}{2}) |\beta\rangle\]
    \[|x\rangle \stackrel{Orcale}\longrightarrow  \cos(\frac{\theta}{2}) |\alpha\rangle - \sin(\frac{\theta}{2}) |\beta\rangle\]
    \[\cos(\frac{\theta}{2}) |\alpha\rangle - \sin(\frac{\theta}{2}) |\beta\rangle\stackrel{Combined}\longrightarrow \left(2|\psi\rangle\langle\psi| - I\right) \left(\cos(\frac{\theta}{2}) |\alpha\rangle - \sin(\frac{\theta}{2}) |\beta\rangle \right)\]
    \[\longrightarrow 
    \begin{bmatrix}
        2\cos^2(\frac{\theta}{2}) - 1 & 2\cos(\frac{\theta}{2})\sin(\frac{\theta}{2}) \\
        2\cos(\frac{\theta}{2})\sin(\frac{\theta}{2}) & 2\sin^2(\frac{\theta}{2}) - 1
    \end{bmatrix}
    \begin{bmatrix}
        \cos(\frac{\theta}{2})|\alpha\rangle \\
        -\sin(\frac{\theta}{2})|\beta\rangle
    \end{bmatrix}
    \]
    \[\longrightarrow
        \cos(\frac{3\theta}{2}) |\alpha\rangle + \sin(\frac{3\theta}{2}) |\beta\rangle
    \]

    证明确实向$|\beta\rangle$方向旋转了$\theta$角度。
    
    \item 归纳假设证明
    假设$G$算子作用到第k-1次时成立:
    \[|x\rangle = \cos(\frac{(2k-1)\theta}{2}) |\alpha\rangle + \sin(\frac{(2k-1)\theta}{2}) |\beta\rangle\]
    现证明$G$算子作用到第k次时成立:
    \[|x\rangle = \cos(\frac{(2k-1)\theta}{2}) |\alpha\rangle + \sin(\frac{(2k-1)\theta}{2}) |\beta\rangle\]
    \[|x\rangle \stackrel{Orcale}\longrightarrow  \cos(\frac{(2k-1)\theta}{2}) |\alpha\rangle - \sin(\frac{(2k-1)\theta}{2}) |\beta\rangle\]    
    \[ \cos(\frac{(2k-1)\theta}{2}) |\alpha\rangle - \sin(\frac{(2k-1)\theta}{2}) |\beta\rangle\]
    \[\stackrel{Combined}\longrightarrow \left(2|\psi\rangle\langle\psi| - I\right) \left(\cos(\frac{(2k-1)\theta}{2}) |\alpha\rangle - \sin(\frac{(2k-1)\theta}{2}) |\beta\rangle \right)\]
    \[\longrightarrow 
    \begin{bmatrix}
        2\cos^2(\frac{\theta}{2}) - 1 & 2\cos(\frac{\theta}{2})\sin(\frac{\theta}{2}) \\
        2\cos(\frac{\theta}{2})\sin(\frac{\theta}{2}) & 2\sin^2(\frac{\theta}{2}) - 1
    \end{bmatrix}
    \begin{bmatrix}
        \cos(\frac{(2k-1)\theta}{2})|\alpha\rangle \\
        -\sin(\frac{(2k-1)\theta}{2})|\beta\rangle
    \end{bmatrix}
    \]
    \[\longrightarrow
        \cos(\frac{(2k+1)\theta}{2}) |\alpha\rangle + \sin(\frac{(2k+1)\theta}{2}) |\beta\rangle
    \]
    证明确实向$|\beta\rangle$方向旋转了$\theta$角度。
\end{enumerate}
根据数学归纳法,我们证明了Grover算法中的算子$G$每次作用时使量子态向$|\beta\rangle$方向旋转角度$\theta$。

\vskip 0.3 in

{\bf 3.} 根据 Grover 算法中 $M$、$N$ 的定义,令 $\gamma = M/N$,证明在 $|\alpha\rangle$、$|\beta\rangle$ 基下,Grover 算法中的算子 $G$ 可以写为 $\begin{bmatrix}
    1-2\gamma & -2\sqrt{\gamma-\gamma^2} \\ 2\sqrt{\gamma-\gamma^2} & 1-2\gamma
\end{bmatrix}$。

由上一道题的证明,我们知道Grover算法中的算子$G$可以分为两步:
\begin{enumerate}[i.]
    \item Orcale
    \[|x\rangle = p |\alpha\rangle + q |\beta\rangle \stackrel{Orcale}\longrightarrow  p |\alpha\rangle - q|\beta\rangle\]
    \item Combined
    \[|x\rangle = p |\alpha\rangle + q |\beta\rangle =  \left(2|\psi\rangle\langle\psi| - I\right) \left(p |\alpha\rangle + q |\beta\rangle \right)\]
\end{enumerate}
我们将两个酉矩阵相乘,即:
\[
\begin{bmatrix}
    1 & 0\\
    0 & -1
\end{bmatrix}
\times
\begin{bmatrix}
    2\cos^2(\frac{\theta}{2}) - 1 & 2\cos(\frac{\theta}{2})\sin(\frac{\theta}{2}) \\
    2\cos(\frac{\theta}{2})\sin(\frac{\theta}{2}) & 2\sin^2(\frac{\theta}{2}) - 1
\end{bmatrix}
=
\begin{bmatrix}
    2cos(\frac{\theta}{2})^2 - 1 & 2\cos(\frac{\theta}{2})\sin(\frac{\theta}{2}) \\
    -2\cos(\frac{\theta}{2})\sin(\frac{\theta}{2}) & 1-2\sin^2(\frac{\theta}{2})
\end{bmatrix}
\]
\[
= 
\begin{bmatrix}
    1-2sin^2(\frac{\theta}{2}) & -2\cos(\frac{\theta}{2})\sin(\frac{\theta}{2}) \\
    2\cos(\frac{\theta}{2})\sin(\frac{\theta}{2}) & 1-2\sin^2(\frac{\theta}{2})
\end{bmatrix}
\]
在上一个问题中我们同样表示了$\theta$,即:
$\sqrt{\frac{N-M}{N}}|\alpha\rangle + \sqrt{\frac{M}{N}} |\beta\rangle= \cos(\frac{\theta}{2}) |\alpha\rangle + \sin(\frac{\theta}{2}) |\beta\rangle$
\[\therefore sin(\frac{\theta}{2}) = \sqrt{\frac{M}{N}} = \sqrt{\gamma},cos(\frac{\theta}{2}) = \sqrt{\frac{N-M}{N}} = \sqrt{1-\gamma} \]

\[\therefore G 
=
\begin{bmatrix}
    1-2sin^2(\frac{\theta}{2}) & -2\cos(\frac{\theta}{2})\sin(\frac{\theta}{2}) \\
    2\cos(\frac{\theta}{2})\sin(\frac{\theta}{2}) & 1-2\sin^2(\frac{\theta}{2})
\end{bmatrix}
=
\begin{bmatrix}
    1-2\gamma & -2\sqrt{\gamma-\gamma^2} \\
    2\sqrt{\gamma-\gamma^2} & 1-2\gamma
\end{bmatrix}
证毕。

\vskip 0.3 in

{\bf Bonus:} 给出 RSA 算法加密、解密过程的证明,即证明明文为 $a \equiv C^d \mod n$。

\textbf{证明目标:} $a \equiv C^d \mod n$。

\textbf{已知条件:}
\begin{enumerate}[(1)]
    \item 加密过程产生等式
        \begin{enumerate}[i.]
            \item 明文 $a$,密文 $C$,且 $0 \leq a < n$,因为解密一直得到小于n的值。
            \item $n = pq$,其中$p,q$为素数。
            \item $\phi(n) = (p-1)(q-1)$。
            \item $e$满足$1 < e < \phi(n)$且$\gcd(e, \phi(n)) = 1$。
            \item $d \equiv e^{-1} \mod \phi(n)$。
            \item $C = a^e \mod n$。
        \end{enumerate}
    \item 数学定理
    \begin{enumerate}[i.]
        \item  Theorem1\_欧拉函数性质:对于任意素数$p,q$,$n = pq$,有 $\phi(n) = (p-1)(q-1)$。
        \item  Theorem2\_欧拉定理:对于任意整数 $a$ 和 $n$ 互质的情况,有 $a^{\phi(n)} \equiv 1 \mod n$。
        \item  Theorem3\_模逆元:$e,\phi(n)$ 互质,存在整数 $d$ 使得 $ed \equiv 1 \mod \phi(n)$。
    \end{enumerate}
\end{enumerate}
\textbf{证明:}
    \[
    C^d = (a^e)^d = a^{ed} 
    \]
    \[\because Theorem3: \ d \equiv e^{-1} \mod \phi(n) \]
    $ \therefore $存在整数k:
    \[ C^d = a^{ed} = a^{1+k\phi(n)} = a \times (a^{\phi(n)})^k  \]
    分情况讨论:
    \begin{enumerate}[i.]
        \item a,n 互质,根据Theorem2,$a^{\phi(n)} \equiv 1 \mod n  $。
        
            \begin{align}
                C^d = a \times (a^{\phi(n)})^k \equiv a \times \left( (a^{\phi(n)})^k  \mod n \right)  \equiv a \times 1 \equiv a \mod n 
            \end{align}
        \item a,n 不互质
        
        $n = pq$,则说明$a$必然是$p,q$中一个的倍数且不是两者乘积$n$的倍数

        不妨假设a是p的倍数,即$\exists t \in \mathbb{N},\ a = p \times t$
        \[\because Theorem2\]
        \[\therefore a^{\phi(q)} \equiv 1 \mod q \]
        两边同时乘方操作$a^{k\phi(p)}$,即:
        \[a^{k\phi(p)\phi(q)} \equiv 1 \mod q \]
        \[\because Theorem1 \phi(n) = (p-1)(q-1)\]
        \[ \therefore \exists m\in \mathbb{N}\ s.t.\ a^{k\phi(p)\phi(q)} = a^{k\phi(n)} = q \times m +1 \]
        \[ \therefore C^d = a^{1+k\phi(n)} = a \times (q \times m +1) = a+a\times q\times m = a+a\times tm(p\times q) =a+a\times tmn  \]
        \[\therefore C^d \equiv a \mod n \]
    \end{enumerate}
    综上所述,$a \equiv C^d \mod n$。
\end{document}
