\documentclass[11pt]{article}
\usepackage{amsmath,amssymb,enumitem,algorithm,algpseudocode}
\usepackage[UTF8]{ctex}
\usepackage[braket]{qcircuit}
\parindent=22pt
\parskip=3pt
\oddsidemargin 18pt \evensidemargin 0pt
\leftmargin 1.5in
\marginparwidth 1in \marginparsep 0pt \headsep 0pt \topskip 20pt
\textheight 225mm \textwidth 148mm
\renewcommand{\baselinestretch}{1.15}
\begin{document}
\title{{\bf 理论作业一 \quad 量子比特与量子门}}
\author{王晓宇 \quad 3220104364}
\date{\today}
\maketitle

\begin{tabular*}{13cm}{r}
\hline
\end{tabular*}

\vskip 0.3 in

{\bf 1.} 已知双量子比特系统的量子态如下 $|\psi\rangle = \begin{bmatrix} \frac{1}{2} & x & 3x & \frac{i}{2\sqrt{2}} \end{bmatrix} ^ \intercal \in \mathbb{C}^4$ ,求该系统处于 $|01\rangle$ 态的概率。

由归一化条件可知:
$$\left | \frac{1}{2} \right |^2 + \left | x \right |^2 + \left | 3x \right |^2 + \left | \frac{i}{2\sqrt{2}}  \right |^2 = 1$$
$$\therefore \left| x \right|^2 = 1/16$$
$$ p = \alpha_{01}^2 = \left| x \right|^2 = \frac{1}{16}$$

\vskip 0.3 in

{\bf 2.} 已知单量子比特的态矢量为 $|\psi\rangle = \begin{bmatrix} 3/5 \\ 4/5 \end{bmatrix}$ ,求该量子比特的Bloch球坐标。

$$|\psi\rangle = \alpha_{0}|0\rangle + \alpha_{1}|1\rangle = cos\left(\frac{\theta}{2}\right)|0\rangle + e^{i\phi}sin\left(\frac{\theta}{2}\right)|1\rangle$$
$$\therefore cos\left(\frac{\theta}{2}\right) = \frac{3}{5}$$
$$\therefore \theta = \pm 2arccos(\frac{3}{5})$$
\[ \because 0 \le \theta \le \pi \] 
\[ \therefore \theta =  2arccos(\frac{3}{5})  \]

\[ \because sin\left( \frac{\theta}{2}\right) >0 \ and \ \alpha_{1} \ is \ a \ real \]
$$\therefore \phi = 0 $$

因此,Bloch球坐标为:
\[ (\theta, \phi) = \left(2 \arccos\left(\frac{3}{5}\right), 0\right) \]


\vskip 0.3 in

{\bf 3.} Bell 态指双量子比特系统的四个特殊量子态,他们是双量子比特系统中纠缠度最高的量子态,因此也称为最大纠缠态,在量子隐形传态、量子算法中有着广泛的应用。一般而言,Bell 态定义如下:
\begin{align}
    |\beta_{xy}\rangle = \frac{|0y\rangle + (-1)^x|1\bar{y}\rangle}{\sqrt{2}}, \quad x,y \in \{0,1\}, \quad \bar{y} = 1-y
\end{align}
\begin{enumerate}[label=\alph*.]
\item 证明 Bell 态是纠缠态。

When $x = 0 \& y = 0$:
\[  |\beta_{xy}\rangle = \frac{|00\rangle + |11\rangle}{\sqrt{2}}\]
假设这个双量子比特系统可以分解成两个单量子比特的张量积:
\[
|\beta_{xy}\rangle =
 \begin{bmatrix}
a \\
b
\end{bmatrix}
\otimes 
\begin{bmatrix}
c \\
d
\end{bmatrix}
= 
\begin{bmatrix}
ac \\
ad \\
bc \\
bd
\end{bmatrix}
= 
\begin{bmatrix}
\frac{1}{\sqrt{2}} \\
0 \\
0 \\
\frac{1}{\sqrt{2}}
\end{bmatrix}
\]
$\because ad = 0 \& bc = 0$ 

说明a和c必有一个为零,b和d必有一个为零,那么ac和bd必为0,与得到的Bell态不符,所以Bell态不可分。

同理可证以下三个Bell态也是纠缠态。

When $x = 0 \& y = 1$:
\[  |\beta_{xy}\rangle = \frac{|01\rangle + |10\rangle}{\sqrt{2}}\]

When $x = 1 \& y = 0$:
\[  |\beta_{xy}\rangle = \frac{|00\rangle - |11\rangle}{\sqrt{2}}\]

When $x = 1 \& y = 1$:
\[  |\beta_{xy}\rangle = \frac{|01\rangle - |10\rangle}{\sqrt{2}}\]

\item 用 H、X、Z 和 CNOT 门设计四个量子电路,使得初态为 $|00\rangle$ 的双量子比特系统经这些量子电路作用后分别演化为四个 Bell 态。

When $x = 0 \& y = 0$:
\[  |\beta_{xy}\rangle = \frac{|00\rangle + |11\rangle}{\sqrt{2}}\]
\[ \Qcircuit @C=1.0em @R=1.5em {
\lstick{q_0} & \gate{H} & \ctrl{1} & \qw \\
\lstick{q_1} & \qw & \targ & \qw 
} \]


When $x = 0 \& y = 1$:

\[  |\beta_{xy}\rangle = \frac{|01\rangle + |10\rangle}{\sqrt{2}}\]
\[ \Qcircuit @C=1.0em @R=1.5em {
\lstick{q_0} & \gate{H} & \ctrl{1}  & \qw \\
\lstick{q_1} & \gate{X} & \targ  & \qw 
} \]

When $x = 1 \& y = 0$:
\[  |\beta_{xy}\rangle = \frac{|00\rangle - |11\rangle}{\sqrt{2}}\]
\[ \Qcircuit @C=1.0em @R=1.5em {
    \lstick{q_0} & \gate{X} & \gate{H} & \ctrl{1} & \qw \\
    \lstick{q_1} & \qw      & \qw & \targ & \qw 
} \]

When $x = 1 \& y = 1$:
\[  |\beta_{xy}\rangle = \frac{|01\rangle - |10\rangle}{\sqrt{2}}\]
\[ \Qcircuit @C=1.0em @R=1.5em {
\lstick{q_0} & \gate{X} & \gate{H} & \ctrl{1}  & \qw \\
\lstick{q_1} & \gate{X}  & \qw     & \targ  & \qw 
} \]

\end{enumerate}

\vskip 0.3 in

{\bf 4.} 证明下图中的两个量子电路等价。(提示:计算两个量子电路对应的酉矩阵)

\[ \Qcircuit @C=1.0em @R=2.0em {
\lstick{q_0} & \targ & \qw \\
\lstick{q_1} & \ctrl{-1} & \qw
} \]

\[ \Qcircuit @C=1.0em @R=1.5em {
\lstick{q_0} & \gate{H} & \ctrl{1} & \gate{H} & \qw \\
\lstick{q_1} & \gate{H} & \targ & \gate{H} & \qw 
} \]

Answer:
For the first circuit:
$$
\begin{bmatrix}
1 & 0 & 0 & 0 \\
0 & 0 & 0 & 1 \\
0 & 0 & 1 & 0 \\
0 & 1 & 0 & 0
\end{bmatrix}
$$
For the second circuit:
$$
(H \otimes H)CNOT(H \otimes H) = 
$$
$$
\left(\frac{1}{\sqrt{2}}      
\begin{bmatrix}
1 & 1\\
1 & -1 
\end{bmatrix}
\otimes
\frac{1}{\sqrt{2}}
\begin{bmatrix}
1 & 1\\
1 & -1
\end{bmatrix}
\right)
\cdot
\begin{bmatrix}
1 & 0 & 0 & 0 \\
0 & 1 & 0 & 0 \\
0 & 0 & 0 & 1 \\
0 & 0 & 1 & 0
\end{bmatrix}
\cdot
\left(
\frac{1}{\sqrt{2}}      
\begin{bmatrix}
1 & 1\\
1 & -1 
\end{bmatrix}
\otimes
\frac{1}{\sqrt{2}}
\begin{bmatrix}
1 & 1\\
1 & -1
\end{bmatrix}
\right)
=
$$
$$
\frac{1}{4}
\begin{bmatrix}
1 & 1 & 1 & 1 \\
1 & -1 & 1 & -1 \\
1 & 1 & -1 & -1 \\
1 & -1 & -1 & 1
\end{bmatrix}
\cdot
\begin{bmatrix}
1 & 0 & 0 & 0 \\
0 & 1 & 0 & 0 \\
0 & 0 & 0 & 1 \\
0 & 0 & 1 & 0
\end{bmatrix}
\cdot
\begin{bmatrix}
    1 & 1 & 1 & 1 \\
    1 & -1 & 1 & -1 \\
    1 & 1 & -1 & -1 \\
    1 & -1 & -1 & 1
\end{bmatrix} = 
$$
$$
\begin{bmatrix}
    1 & 0 & 0 & 0 \\
    0 & 0 & 0 & 1 \\
    0 & 0 & 1 & 0 \\
    0 & 1 & 0 & 0
    \end{bmatrix}
$$
Since of the two matrices are equal, the two circuits are equivalent.

\vskip 0.3 in

{\bf 5.} 证明厄米算符 $A$ 的任一本征值均为实数,且不同本征值对应的本征态正交。

\begin{enumerate}[label=\alph*.]
\item 先证明厄米算符 $A$ 的任一本征值均为实数:

假设$A$是厄米算符,即
\begin{align}
    A = A^{\dagger}    
\end{align}
我们任取$A$的本征值$\lambda$,有
\begin{align}
    A|\phi\rangle = \lambda|\phi \rangle    
\end{align}
取共轭转置我们有
\[\langle \phi | A^{\dagger}  = \lambda^{*}\langle \phi | \]
由(2)式我们有
\[\langle \phi | A  = \lambda^{*}\langle \phi | \]
同时右乘$|\phi \rangle$得到
\begin{align}
    \langle \phi | A|\phi \rangle=\lambda^{*}\langle \phi |\phi \rangle    
\end{align}
由(3)式我们代换(4)式中的$A|\phi \rangle$得到
\[ \lambda \langle \phi |\phi \rangle=\lambda^{*}\langle \phi |\phi \rangle \]   
由于$\langle \phi |\phi \rangle \neq 0$
\[\therefore \lambda = \lambda^{*}\]
说明本征值为实数

\item 我们接着证明不同本征值对应的本征态正交:

任取$A$的两个本征值$\lambda_1$和$\lambda_2$,对应的本征态分别为$|\phi_1\rangle$和$|\phi_2\rangle$,有
\begin{align}
    A|\phi_1\rangle = \lambda_1|\phi_1 \rangle
\end{align}
\begin{align}
    A|\phi_2\rangle = \lambda_2|\phi_2 \rangle
\end{align}
取(5)式,左乘$\langle \phi_2|$,(6)式左乘$\langle \phi_1|$得到:
\begin{align}
    \langle \phi_2 | A|\phi_1 \rangle = \lambda_1\langle \phi_2 |\phi_1 \rangle
\end{align}
\begin{align}
    \langle \phi_1 | A|\phi_2 \rangle = \lambda_2\langle \phi_1 |\phi_2 \rangle
\end{align}
(7)式取共轭转置得到:
\begin{align}
    \langle \phi_1 | A|\phi_2 \rangle = \langle \phi_1 | A^{\dagger}|\phi_2 \rangle = \lambda_1^{*}\langle \phi_1 |\phi_2\rangle = \lambda_1\langle \phi_1 |\phi_2\rangle
\end{align}
由(8)和(9)式我们有:
\[\lambda_1\langle \phi_2 |\phi_1 \rangle = \lambda_2\langle \phi_1 |\phi_2 \rangle\]
即:
\[ \left(\lambda_1-\lambda_2\right)\langle \phi_1 |\phi_2 \rangle = 0\]
\[\because \lambda_1 \neq \lambda_2\]
\[\therefore \langle \phi_1 |\phi_2 \rangle  = 0\]
即不同本征值对应的本征态正交。
\end{enumerate}
\vskip 0.3 in

{\bf 6.} Deutsch 算法展示了量子计算机强大的并行计算能力。Deutsch-Jozsa 算法是其推广形式,将可分类的函数推广至多比特情形。

已知函数 $f:\{0,1\}^n \rightarrow \{0,1\}$,该函数是常数函数(对所有输入均输出 $0$ ,或对所有输入均输出 $1$)或平衡函数(对恰好一半的输入输出 $0$ ,对另一半输入输出 $1$)。Deutsch-Jozsa 算法只需对实现函数 $f$ 的结构进行一次查询,即可判断 $f$ 是常数函数还是平衡函数。

下图是实现 Deutsch-Jozsa 算法的量子线路。其中,$U_f:|x,y\rangle \to |x,y\oplus f(x)\rangle$ 是实现函数 $f$ 的 $n+1$ 比特的量子门。

\[ \Qcircuit @C=1.0em @R=.7em {
\lstick{\ket{0}} & {/^n} \qw & \gate{H^{\otimes n}} & \multigate{1} {U_f} & \gate{H^{\otimes n}} & \meter & \qw \\
\lstick{\ket{1}} & \qw & \gate{H} & \ghost{U_f} & \qw & \qw & \qw \\
} \]


推导该量子电路中量子态的演化过程,并说明如何基于测量结果判断 $f$ 是常数函数还是平衡函数。(提示:计算 $f$ 为常数函数或平衡函数时的测量结果)


先标注量子电路时间:
\[ \Qcircuit @C=1.0em @R=.7em {
\lstick{\ket{0}} & {/^n} \qw & \gate{H^{\otimes n}} & \multigate{1} {U_f} & \gate{H^{\otimes n}} & \meter & \qw \\
\lstick{\ket{1}} & \qw & \gate{H} & \ghost{U_f} & \qw & \qw & \qw \\}
\]
\[
    \uparrow  \qquad \uparrow   \qquad \quad \uparrow  \quad \qquad \uparrow  \qquad
\]
\[
    \varphi_1  \qquad \varphi_2   \qquad \quad \varphi_3  \qquad \varphi_4  \qquad
\]

假设$f$ 为常数函数时的测量结果:
\begin{enumerate}
    \item $\varphi_1$时刻
    我们引入了$n+1$位bit的量子电路,其中除最低位外,均为$|0\rangle$态,我们这里暂叫这n位量子比特为$x_{\varphi_1}^{\otimes n}$
    \item $\varphi_2$时刻
    \[ |x^{\otimes n}\rangle \stackrel{{H^{\otimes n}}}\longrightarrow \left(\frac{1}{\sqrt{2}}\left(|0\rangle + |1\rangle \right)\right)^{\otimes n} \]
    \[\therefore |x_{\varphi_2}^{\otimes n}\rangle =\left(\frac{1}{\sqrt{2}}\left(|0\rangle + |1\rangle \right)\right)^{\otimes n}\]
    \[|y\rangle = \frac{1}{\sqrt{2}}\left(|0\rangle - |1\rangle \right) \]
    \item $\varphi_3$时刻,在完成$U_f$的函数处理后:
    \[|x_{\varphi_3}^{\otimes n}y\rangle =  |x_{\varphi_2}^{\otimes n}\rangle | \left(y \oplus f\left(x_{\varphi_2}^{\otimes n}\right)\right)\rangle \\
    = \left(-1\right)^{f\left(x_{\varphi_2}^{\otimes n}\right)}|x_{\varphi_2}^{\otimes n} y\rangle 
    \]
    $\because \ f$为常数函数,分类讨论可以得到:
    \begin{enumerate}
        \item $f(x) = 0$
           \[ |x_{\varphi_3}^{\otimes n}y\rangle = |x_{\varphi_2}^{\otimes n}y\rangle = \left(\frac{1}{\sqrt{2}}\left(|0\rangle + |1\rangle \right)\right)^{\otimes n} \left(\frac{1}{\sqrt{2}}\left(|0\rangle - |1\rangle \right)\right)\]
        \item $f(x) = 1$
           \[ |x_{\varphi_3}^{\otimes n}y\rangle = - |x_{\varphi_2}^{\otimes n}y\rangle = - \left(\frac{1}{\sqrt{2}}\left(|0\rangle + |1\rangle \right)\right)^{\otimes n} \left(\frac{1}{\sqrt{2}}\left(|0\rangle - |1\rangle \right)\right)\]
    \end{enumerate}
    
    \[\therefore |x_{\varphi_3}^{\otimes n}y\rangle = \pm \left(\frac{1}{\sqrt{2}}\left(|0\rangle + |1\rangle \right)\right)^{\otimes n} \left(\frac{1}{\sqrt{2}}\left(|0\rangle - |1\rangle \right)\right)\]
    
    \item $\varphi_4$时刻,我们先不看系数以及$|y\rangle$
        \[ \left(\frac{1}{\sqrt{2}}\left(|0\rangle + |1\rangle \right)\right)^{\otimes n} \stackrel{{H^{\otimes n}}}\longrightarrow  |0\rangle^{\otimes n} \]
        \[\therefore  |x_{\varphi_4}^{\otimes n}y\rangle = |0\rangle^{\otimes n} \]
        \[\therefore |x_{\varphi_4}^{\otimes n}y\rangle = \pm |0\rangle^{\otimes n} \left(\frac{1}{\sqrt{2}}\left(|0\rangle - |1\rangle \right)\right)\]
\end{enumerate}
所以我们可以通过测量n位比特是否全为$|0\rangle$来判断$f$是常数函数还是平衡函数,如果全为0,则$f$为常数函数,否则为平衡函数。

\end{document}

